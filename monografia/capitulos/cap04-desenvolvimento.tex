\chapter{DESENVOLVIMENTO}
\label{cap:desenvolvimento}

Este capítulo apresenta o desenvolvimento do projeto fundamentado em duas metodologias consolidadas: Design Thinking e Scrum. A primeira orienta a progressão desde a compreensão profunda do problema (Imersão), passando pela estruturação de soluções (Ideação), até a materialização tecnológica (Prototipação). A segunda, aplicada especificamente na fase de Prototipação, organiza o trabalho técnico em entregas iterativas que permitiram a validação contínua das funcionalidades pelos stakeholders.

\section{Imersão}
\label{sec:imersao}

A etapa da imersão do Design Thinking, segundo Vianna et al. (2012), é o momento em que uma equipe de projeto busca aproximação com o contexto de um problema latente, identificando necessidades e oportunidades que nortearão a geração de soluções. Assim, esta fase do projeto teve como objetivo compreender os fundamentos e contexto do evento, bem como seus processos operacionais e os desafios enfrentados pela equipe. Afim de garantir uma boa imersão e ideação da situação atual, bem como identificar as lacunas atuais que necessitam de inovação tecnológica, foi aplicado um questionário estruturado à Comissão Organizadora do evento Descida da Ladeira, entre os dias 20 a 25 de outubro de 2025, totalizando 6 inquiridos entre professores e membros da organização com diferentes níveis de envolvimento e experiência com o evento.

\subsection{Contextualização do Evento}
\label{subsec:contextualizacao-evento}

O evento Descida da Ladeira teve sua primeira edição realizada em fevereiro de 2014, por iniciativa do Prof. Fioravante, com o objetivo de criar um momento de integração entre alunos de diversos cursos que estudavam em períodos diferentes. Ao longo de sua trajetória, o evento consolidou-se como projeto de extensão de caráter pedagógico na Fatec Mogi Mirim, tendo realizado um total de 10 edições até 2025, com uma interrupção motivada pela pandemia de COVID-19. Segundo os dados coletados, o evento mantém uma média consistente de 25 equipes participantes por edição, variando entre 20 e 30 competidores conforme a mobilização em cada ano. A competição envolve carrinhos de rolimã projetados e construídos pelos próprios estudantes, que descem a ladeira central da instituição em baterias eliminatórias até a definição dos vencedores, sendo facultada a participação de alunos de todos os cursos oferecidos pela unidade. O processo operacional do evento inicia-se na fase pré-competição com o sistema de inscrições, atualmente realizado através de formulário digital na plataforma Google Forms. Esta mudança do formato físico para o digital ocorreu durante o período da pandemia e foi mantida pelas vantagens operacionais observadas. No formulário, são coletadas informações essenciais como nome da equipe, identificação de um piloto responsável, identificação de um "motor" (membro responsável pelo empurrão inicial), relação completa dos membros da equipe com seus respectivos cursos e períodos, além da possibilidade de upload de uma imagem que servirá como logotipo oficial da equipe durante o evento. Paralelamente ao sistema de inscrições, é disponibilizado online através do site institucional um regulamento detalhado contendo todas as regras técnicas de segurança, especificações dos carrinhos, critérios de desclassificação e funcionamento das baterias.

\subsection{Fluxo do Evento}
\label{subsec:fluxo-evento}

No dia do evento, realizado tipicamente das 8h00 às 13h00, o fluxo operacional inicia-se com a organização dos espaços e demarcação precisa da pista de competição. A equipe organizadora distribui os diversos postos necessários ao longo do percurso, desde a linha de largada até a linha de chegada na ladeira central da Fatec. Após a liberação oficial da pista, realiza-se uma vistoria técnica detalhada de todos os carrinhos inscritos, verificando requisitos de segurança, conformidade com o regulamento e condições gerais dos equipamentos. Concluída a vistoria, inicia-se a fase de tomadas de tempo classificatórias, onde cada equipe realiza entre 2 e 3 descidas cronometradas. As baterias são organizadas com grupos de 3 carrinhos descendo simultaneamente, permitindo comparação direta de desempenho. Dependendo do número total de equipes participantes, o formato de eliminação pode incluir uma fase intermediária de classificação antes da final, onde são selecionados os três melhores tempos acumulados para disputar o título. A bateria final é disputada pelos três finalistas em descida única, sendo declarado vencedor aquele que cruzar primeiro a linha de chegada. O processo de cronometragem, elemento central e crítico da competição, ocorre de forma distribuída e manual entre dois pontos principais do percurso. Na linha de largada, posicionam-se dois fiscais de pista responsáveis por garantir o posicionamento correto dos carrinhos e acionar os dispositivos de sinalização. A largada oficial é marcada simultaneamente por uma buzina sonora e pelo abaixamento de uma bandeira, ambos operados manualmente pelos fiscais. A meio do trajeto, encontra-se o fiscal de bandeira, responsável por liberar visualmente a pista para o público. Na linha de chegada, um fiscal munido de cronômetro digital (tipicamente um smartphone com aplicativo de cronometragem) assume a função crítica de registrar os tempos. Ao ouvir a buzina de largada, este fiscal inicia manualmente o cronômetro e, à medida que cada carrinho cruza a linha de chegada, registra os três tempos individuais utilizando um cronômetro de múltiplos estágios. Os tempos capturados são então anotados em papel ou mantidos na tela do dispositivo e imediatamente transmitidos via aplicativo WhatsApp para a equipe de controle, responsável pelo lançamento manual dos dados em planilha Excel, cálculo de médias quando aplicável e elaboração da classificação atualizada para divulgação.

\subsection{Infraestrutura do Evento}
\label{subsec:infraestrutura-evento}

Quanto à infraestrutura disponível para realização do evento, identificou-se uma situação de recursos limitados mas adaptáveis. A área da ladeira utilizada para a competição não dispõe de infraestrutura elétrica permanente, sendo necessário adaptar ligações provisórias a partir de postes próximos especificamente no dia do evento. Esta adaptação fornece energia suficiente para os equipamentos essenciais como televisor, computador e sistema de som, mas apresenta limitações de capacidade e distribuição espacial dos pontos de energia. Um desafio de infraestrutura particularmente crítico é a ausência completa de conectividade Wi-Fi estável na área da ladeira, impossibilitando qualquer solução que dependa de internet durante a competição. A distância entre os pontos de largada e chegada não foi quantificada precisamente nas respostas, mas é suficiente para criar desafios de comunicação e sincronização entre as equipes posicionadas em cada extremidade. Os recursos de comunicação atuais limitam-se a rádios convencionais e sistema de som com microfone para chamamento das equipes e comunicação com o público presente.

\subsection{Problemática}
\label{subsec:problematica}

A partir da análise consolidada das 27 questões aplicadas e suas respectivas respostas, foi possível identificar e categorizar três grupos principais de problemas no processo operacional atual, com diferentes níveis de impacto e criticidade. Esta categorização baseou-se na frequência de menções pelos respondentes, no grau de preocupação expressado e na avaliação dos impactos diretos sobre a qualidade, justiça e fluidez da competição. Os problemas identificados distribuem-se entre questões de precisão técnica na captura de dados, deficiências nos fluxos de comunicação e informação, e limitações operacionais decorrentes da natureza manual e distribuída dos processos atuais. O primeiro e mais crítico problema identificado refere-se ao processo de cronometragem manual, apontado por todos os respondentes como fonte de preocupação quanto à precisão e confiabilidade dos resultados. A captura do tempo de chegada depende inteiramente da reação humana do fiscal, que deve observar visualmente o momento exato em que cada carrinho cruza a linha e acionar manualmente o cronômetro. Esta dependência do fator humano introduz uma margem de erro inerente devido ao tempo de reação, variações de atenção e diferenças de interpretação sobre o momento exato da passagem pela linha. Conforme relatado pelos respondentes, o processo torna-se "trabalhoso e repetitivo", especialmente considerando que cada bateria envolve três carrinhos e o evento pode ter mais de 20 equipes realizando múltiplas descidas. A incerteza de medição resultante cria uma situação onde diferenças de décimos ou mesmo centésimos de segundo - potencialmente decisivas para a classificação - podem ser questionadas pelos competidores, gerando contestações e comprometendo a percepção de justiça do processo competitivo. Um respondente sintetizou a questão afirmando que "a medição do tempo ocorre ainda de maneira um tanto arcaica, o que destoa do viés tecnológico que a unidade e seus cursos apregoam", evidenciando não apenas uma limitação técnica mas também uma incongruência entre o perfil institucional e os métodos empregados. O segundo problema identificado relaciona-se à latência significativa entre a captura dos tempos na linha de chegada e sua efetiva divulgação para participantes e público. O fluxo atual envolve múltiplas etapas sequenciais: o fiscal de chegada registra os tempos no cronômetro, anota ou fotografa os valores, envia via mensagem de WhatsApp para a equipe de controle, que então realiza o lançamento manual dos dados em planilha Excel, executa cálculos de média quando necessário, atualiza a classificação geral e finalmente projeta os resultados na televisão disponível ao público. Este processo, embora funcional, consome tempo considerável e gera um atraso perceptível entre a conclusão de cada bateria e o conhecimento de seus resultados. Conforme apontado pelos respondentes, "a divulgação de cada corrida geralmente não acontece, somente é divulgado no som de rádio local no dia do evento os participantes que avançaram para próximas fases", e os resultados são mostrados na TV "só após lançados", sem qualquer visualização em tempo real do andamento da competição. Esta latência impacta negativamente a experiência tanto dos competidores, que ficam em suspense aguardando seus resultados, quanto do público, que perde o dinamismo de acompanhar instantaneamente o desempenho de cada equipe. Adicionalmente aos problemas de cronometragem e divulgação, identificou-se uma deficiência significativa na comunicação operacional com as equipes participantes durante o evento. O sistema atual baseia-se em chamamento via microfone e rádio para convocar as equipes que devem se posicionar para a próxima bateria. Entretanto, conforme relatado explicitamente por um dos respondentes, "frequentemente os membros não ouvem o chamamento no microfone", gerando atrasos na sequência das baterias, necessidade de chamamentos repetidos e eventual desorganização do cronograma do evento. Este problema de comunicação contribui para as dificuldades de "gestão de tempo" citadas como um dos maiores desafios operacionais do dia da competição. A natureza difusa do áudio em ambiente aberto, combinada com a dispersão natural das equipes pela área do evento enquanto aguardam sua vez, cria uma situação onde a responsabilidade de estar atento aos chamamentos recai inteiramente sobre os participantes, sem um mecanismo confiável de notificação ou confirmação de recebimento da convocação.

\subsection{Integrações Tecnológicas Anteriores}
\label{subsec:integracoes-anteriores}

O levantamento também revelou um histórico de tentativas anteriores de incorporação de tecnologia ao evento, com resultados mistos que fornecem lições importantes para o projeto atual. Segundo os respondentes, duas iniciativas tecnológicas foram implementadas em edições passadas: um sistema de sirene e luzes para sinalização de largada, que "funcionou bem" e permanece em uso até hoje; e uma tentativa de cronometragem automática baseada em um pórtico eletrônico instalado na linha de chegada, que "não funcionou bem" e foi descontinuada. Adicionalmente, foi mencionado o sistema desenvolvido no TCC de Eugênio realizado em 2025, embora os respondentes não tenham fornecido detalhes sobre suas funcionalidades específicas ou resultados de uso, indicando que o conhecimento sobre este sistema não está uniformemente distribuído entre a comissão organizadora ou que sua implementação foi limitada. O uso atual de Excel para lançamento de resultados e TV para divulgação representa a linha de base tecnológica que tem sido funcional, ainda que com as limitações já descritas de latência e trabalho manual intensivo. A experiência malsucedida com a tentativa de automação completa via pórtico eletrônico demonstra que soluções excessivamente complexas ou que removem inteiramente a supervisão humana podem não ser adequadas ao contexto específico do evento. As razões para a falha do sistema de pórtico não foram detalhadas pelos respondentes, mas tipicamente sistemas deste tipo enfrentam desafios relacionados à sensibilidade dos sensores, calibração para diferentes condições ambientais (luminosidade, velocidade variável dos carrinhos), necessidade de manutenção técnica especializada e dificuldade de validação imediata em caso de leituras duvidosas. Esta experiência negativa criou, naturalmente, certa cautela entre os organizadores quanto a propostas de automação, como evidenciado pela preocupação expressa sobre a "confiabilidade" de sistemas automatizados. Entretanto, os mesmos respondentes manifestaram entusiasmo com a possibilidade de cronometragem automática com sensores, desde que "funcione" adequadamente, e sugeriram explicitamente que a confiabilidade poderia ser assegurada "mantendo-se em paralelo a medição manual, para conferência". Esta receptividade condicionada indica uma preferência por abordagens híbridas que combinem tecnologia e supervisão humana, maximizando precisão sem eliminar a possibilidade de validação.

\subsection{Propostas de Soluções}
\label{subsec:propostas-solucoes}

A partir da análise consolidada dos dados coletados na fase de imersão e considerando as experiências anteriores com tecnologia no evento, identificou-se que as necessidades prioritárias para o novo sistema concentram-se em três frentes principais que respeitam as severas limitações de infraestrutura identificadas e mantêm conscientemente o elemento humano no processo de validação dos resultados. Esta abordagem alinha-se à lição aprendida com a falha do sistema de pórtico automático, optando por uma estratégia de digitalização híbrida que potencializa as capacidades humanas através da tecnologia ao invés de tentar substituí-las completamente. O escopo definido para o projeto foca, portanto, em resolver os problemas de latência, integração de dados e comunicação operacional, deixando explicitamente de fora a automação completa da cronometragem com sensores, que demandaria infraestrutura e validação além das possibilidades atuais do evento. A primeira necessidade prioritária identificada é a integração da cronometragem manual ao sistema digital, eliminando as etapas intermediárias de transcrição e transmissão que atualmente consomem tempo e introduzem possibilidades de erro. O novo sistema deve permitir que o fiscal de chegada, munido de um dispositivo digital, insira os tempos capturados diretamente no sistema através de uma interface simplificada e intuitiva, adequada ao contexto de uso em ambiente externo e sob a pressão temporal da competição. Esta entrada direta deve substituir imediatamente o fluxo atual de anotação em papel, fotografia de cronômetros, envio via WhatsApp e lançamento manual em Excel, reduzindo drasticamente o número de etapas e a possibilidade de erros de transcrição. É importante ressaltar que esta funcionalidade mantém o aspecto manual e humano da captura do tempo através do cronômetro, validado e aceito ao longo das 10 edições do evento, mas moderniza e agiliza tudo que ocorre após esta captura inicial, transformando um processo analógico-digital fragmentado em um fluxo digital integrado. A segunda necessidade prioritária refere-se à divulgação de resultados em tempo real, eliminando a latência atualmente experimentada entre a conclusão de uma bateria e o conhecimento de seus resultados por participantes e público. O sistema deve expor dados atualizados instantaneamente sobre a classificação geral, resultados da bateria mais recente, próximas equipes a competir e outras informações relevantes. A atualização em tempo real significa que, no momento em que o fiscal de chegada confirma a entrada dos tempos no sistema, estes dados tornam-se imediatamente disponíveis em todas as interfaces conectadas, sem necessidade de processamento manual intermediário. Esta funcionalidade transforma radicalmente a experiência do evento, permitindo que a emoção de cada descida seja acompanhada instantaneamente por seus resultados objetivos. A terceira e mais inovadora necessidade identificada relaciona-se à redução da latência na comunicação entre os pontos operacionais do evento, especificamente entre a linha de chegada e a linha de largada, utilizando tecnologia de comunicação de longo alcance LoRa (Long Range). Atualmente, a comunicação entre estes pontos depende de rádios convencionais sujeitos a interferências, alcance limitado e necessidade de operação manual. O sistema proposto implementará uma rede LoRa onde a estação de chegada, operada pelo cronometrista, terá a capacidade de acionar remotamente a buzina de largada, estabelecendo uma sincronização precisa e eliminando a dependência de coordenação por rádio entre equipes fisicamente distantes. Este acionamento remoto resolve simultaneamente dois problemas: garante que o cronometrista esteja pronto e atento no exato momento da largada (pois é ele próprio quem a aciona), e elimina o delay de comunicação que existe no fluxo atual onde a equipe de largada aciona a buzina e precisa comunicar via rádio o momento exato para a equipe de chegada iniciar a cronometragem. A escolha pela tecnologia LoRa justifica-se pela necessidade de funcionar em ambiente sem Wi-Fi, pelo longo alcance necessário entre os pontos do percurso, pelo baixo consumo de energia que permite operação com baterias portáteis, e pela confiabilidade superior a rádios convencionais em ambientes com múltiplas fontes de interferência. Adicionalmente às três necessidades funcionais prioritárias, os respondentes foram unânimes em destacar requisitos técnicos não-funcionais essenciais que devem permear todas as decisões de arquitetura e implementação do sistema. O primeiro e mais crítico destes requisitos é a capacidade de operação completamente offline, sem qualquer dependência de conectividade com a internet durante o evento. Esta exigência decorre diretamente da constatação de que não há Wi-Fi estável disponível na área da ladeira e de que a conectividade móvel pode ser intermitente ou congestionada devido à concentração de pessoas no local. O sistema deve, portanto, ser arquitetado para funcionar de forma autônoma através de rede local, com todos os componentes comunicando-se diretamente entre si sem necessidade de servidores externos ou serviços em nuvem durante a operação. O segundo requisito essencial refere-se à resiliência e confiabilidade do sistema, que não pode apresentar falhas ou indisponibilidades durante o evento. Diferentemente de sistemas corporativos onde uma falha temporária pode ser contornada, o evento Descida da Ladeira ocorre uma única vez por ano em um período concentrado de aproximadamente 5 horas, não havendo possibilidade de "tentar novamente mais tarde". Por fim, o terceiro requisito crítico é a simplicidade operacional, considerando que o sistema será utilizado por pessoas sob pressão temporal e em ambiente externo, muitas vezes sem treinamento técnico aprofundado. As interfaces devem ser intuitivas, os fluxos de operação devem ser diretos e inequívocos, e o sistema como um todo deve "desaparecer" permitindo que os operadores foquem na condução do evento ao invés de lutar com a tecnologia.

\subsection{Conclusão da fase}
\label{subsec:conclusao-imersao}

Esta fase de imersão forneceu, portanto, uma compreensão abrangente e empiricamente fundamentada do contexto operacional do evento Descida da Ladeira, permitindo direcionar o desenvolvimento do sistema para solucionar problemas reais, prioritários e consensualmente reconhecidos pela equipe organizadora. A abordagem metodológica de coletar dados diretamente dos \emph{stakeholders} através de questionário estruturado, complementada pela análise do histórico de tentativas tecnológicas anteriores, resultou em um conjunto bem definido de requisitos funcionais e não-funcionais que orientarão as fases subsequentes de ideação e prototipação. Mais importante ainda, a imersão evidenciou que a solução tecnológica adequada para este contexto não é necessariamente a mais sofisticada ou automatizada, mas sim aquela que se adapta às limitações reais de infraestrutura, respeita os processos validados ao longo de 10 edições do evento, e foca em eliminar os gargalos específicos que causam os maiores impactos negativos na experiência de organizadores, competidores e público. A próxima fase de ideação partirá destes \emph{insights} para explorar alternativas de arquitetura, justificar escolhas tecnológicas e detalhar como cada componente do sistema proposto atenderá às necessidades identificadas dentro das restrições estabelecidas.

\section{Ideação}
\label{sec:ideacao}

A etapa de Ideação, segundo Vianna et al. (2012), representa o momento de transformar os \emph{insights} coletados na fase de Imersão em oportunidades concretas e soluções inovadoras. Nesta fase, a equipe de projeto analisa os dados levantados, sintetiza os principais desafios identificados e projeta alternativas tecnológicas que respondam às necessidades dos \emph{stakeholders}. Assim, a Ideação constitui a ponte entre a compreensão profunda do problema e a materialização da solução proposta, fundamentando as escolhas arquiteturais e tecnológicas que orientarão o desenvolvimento do sistema. A análise das respostas obtidas no questionário aplicado junto à Comissão Organizadora revelou padrões consistentes que evidenciam tanto as potencialidades quanto às fragilidades do processo atual de gerenciamento do evento Descida da Ladeira. A síntese desses \emph{insights} permitiu identificar cinco categorias principais de oportunidades de melhoria, as quais nortearam a concepção da solução tecnológica. O processo atual de cronometragem, realizado manualmente com cronômetros digitais e registro em planilhas, apresenta-se como o principal gargalo operacional identificado. Os organizadores destacaram a dificuldade em garantir precisão absoluta nos registros, especialmente em situações de múltiplas equipes descendendo simultaneamente ou em rápida sucessão. A transcrição manual dos tempos para planilhas eletrônicas introduz riscos de erro humano e atrasos na divulgação dos resultados, gerando períodos de incerteza para as equipes participantes. Ademais, a ausência de um sistema centralizado dificulta a validação cruzada dos dados e a manutenção de histórico confiável entre edições. A comunicação entre os organizadores ocorre predominantemente através de grupos de WhatsApp e rádios comunicadores durante o evento. Embora funcionais, esses canais apresentam limitações significativas: mensagens importantes podem ser perdidas no fluxo contínuo de conversas, não há garantia de que todos os membros da equipe recebam as informações, e a comunicação via rádio depende de proximidade física e disponibilidade de equipamentos. Os respondentes manifestaram interesse em mecanismos mais eficazes de broadcast de informações críticas, como mudanças de horário, resultados de baterias e convocações para vistoria. A publicação dos resultados atualmente ocorre de forma manual, com considerável intervalo entre o término de cada bateria e a divulgação oficial das classificações. Esse hiato temporal, embora compreensível dadas as limitações do processo manual, reduz o engajamento do público presente e das equipes já eliminadas, que frequentemente deixam o local sem conhecer os resultados finais. Os organizadores reconheceram que a modernização desse aspecto poderia aumentar significativamente a experiência de todos os envolvidos, transformando o evento em algo mais dinâmico e transparente.

\subsection{Definição de Atores e Casos de Uso}
\label{subsec:atores-casos-uso}

A partir dos \emph{insights} sintetizados, procedeu-se à identificação dos perfis de usuário que irão interagir com o sistema proposto, bem como das funcionalidades necessárias para atender suas necessidades específicas. Segundo Pressman e Maxim (2016), a modelagem de casos de uso constitui ferramenta fundamental para capturar requisitos funcionais sob a perspectiva dos usuários, facilitando a comunicação entre \emph{stakeholders} técnicos e não-técnicos. Foram identificados quatro perfis principais de usuários, cada qual com necessidades e níveis de acesso distintos: a) O Organizador (Administrador) representa a Comissão Organizadora do evento, composta por professores e coordenadores responsáveis pela gestão completa. Este perfil possui acesso irrestrito a todas as funcionalidades do sistema, desde a configuração de uma nova edição até a geração de relatórios finais, incluindo a aprovação de inscrições, validação de tempos e gerenciamento de usuários. b) O Cronometrista (Juiz) corresponde ao membro da equipe responsável pela operação da cronometragem durante o evento. Sua interação com o sistema concentra-se nas funcionalidades de registro de tempos e sua validação, exigindo interface otimizada para agilidade e precisão nas operações realizadas sob pressão temporal. c) O Público (Espectador) engloba a comunidade acadêmica e visitantes presentes no evento. Para este perfil, o sistema oferece acesso de visualização aos resultados em tempo real e histórico de corridas, sem necessidade de autenticação, promovendo transparência e engajamento. A modelagem dos casos de uso organizou as funcionalidades do sistema em cinco módulos principais, conforme ilustrado na Figura X. O módulo de Gestão do Evento contempla as funcionalidades administrativas fundamentais: criação de novas edições anuais, configuração da estrutura de baterias (eliminatórias, semifinais e finais), definição de cronograma e geração automática de chaveamento das equipes. O módulo de Gestão de Equipes concentra as funcionalidades relacionadas ao ciclo de vida das inscrições: a importação das inscrições, o processo de aprovação e o registro de vistoria técnica dos carrinhos através de checklist.
\includegraphics[width=4.82543in,height=6.7875in]{vertopal_8b17decc23bf4529b4e7a127829a3ec2/media/image4.png}

\begin{quote}
Figura X - Diagrama de Casos de Uso do Sistema de Gerenciamento do
Evento Descida da Ladeira\\
Fonte: Dos próprios autores (2025).
\end{quote}

O módulo de Cronometragem representa o núcleo operacional do sistema no dia do evento. Engloba a funcionalidade de iniciar corridas (marcando o momento da largada), o registro dos tempos de chegada através de interface otimizada e a validação posterior desses tempos pelos juízes. A decisão de manter a cronometragem manual, ao invés de implementar sensores automáticos, fundamentou-se na análise de custo-benefício e na confiabilidade já estabelecida do processo atual, direcionando os esforços de inovação para a otimização da interface de registro e para a transparência na divulgação dos dados. O módulo de Resultados e Comunicação foca na experiência do usuário final, oferecendo visualização de ranking atualizado em tempo real, consulta ao histórico de tempos de baterias anteriores, geração de relatórios finais oficiais, e exportação de dados para análises externas. A atualização em tempo real do ranking constitui um dos diferenciais principais da solução proposta, eliminando os hiatos temporais atualmente existentes entre a conclusão das corridas e a divulgação dos resultados. Por fim, o módulo de Gestão de Usuários provê as funcionalidades de autenticação no sistema e gerenciamento de permissões, garantindo controle de acesso apropriado conforme o perfil de cada usuário. A estrutura hierárquica de atores, com relações de generalização entre Público, Cronometrista e Organizador, simplifica significativamente o modelo de permissões do sistema. Ao invés de duplicar casos de uso ou criar relacionamentos complexos, a herança de permissões garante que perfis superiores automaticamente possuem capacidades dos perfis inferiores, refletindo naturalmente a realidade operacional onde organizadores frequentemente assumem funções de cronometragem. A ausência de relacionamentos \emph{«include»} e \emph{«extend»} entre casos de uso, comuns em diagramas mais detalhados, justifica-se pelo nível conceitual desta modelagem na fase de Ideação. Tais relacionamentos poderão ser explicitados durante a Prototipação, quando as especificações textuais detalhadas de cada caso de uso revelarão dependências e extensões específicas.

\subsection{Modelagem Conceitual de Dados}
\label{subsec:modelagem-dados}

A estrutura de dados do sistema foi concebida considerando os requisitos funcionais identificados e as necessidades de rastreabilidade e auditoria inerentes a eventos competitivos. O Modelo Entidade-Relacionamento conceitual, apresentado na Figura Y, organiza as informações em cinco agrupamentos lógicos principais. O primeiro agrupamento, Gestão do Evento, estabelece a hierarquia fundamental através das entidades EVENTO e EDIÇÃO. A separação dessas entidades, ao invés de uma estrutura única, permite manter histórico completo de todas as realizações do evento ao longo dos anos, possibilitando análises comparativas e evolução temporal. Cada EVENTO (entidade permanente) possui múltiplas EDIÇÕES (instâncias anuais/semestrais), as quais por sua vez agregam as EQUIPES participantes e as BATERIAS que estruturam a competição. O segundo agrupamento, Participantes, modela a composição das equipes através das entidades EQUIPE, MEMBRO e CARRINHO. Cada EQUIPE é composta por múltiplos MEMBROS (alunos identificados por RA e função - piloto ou mecânico) e possui um único CARRINHO, sobre o qual são registradas as informações de vistoria técnica. A cardinalidade 1:1 entre EQUIPE e CARRINHO reflete a regra de negócio que determina uma equipe por carrinho.
Figura Y - Modelo Entidade-Relacionamento Conceitual do Sistema

Fonte: Dos próprios autores (2025).
O terceiro agrupamento, Competição, estrutura a organização temporal do evento através das entidades BATERIA e CORRIDA. Cada BATERIA (fase eliminatória, semifinal ou final) contém múltiplas CORRIDAS (descidas individuais), estabelecendo a hierarquia que permite gerenciar a progressão do evento desde as eliminatórias até a final. O quarto agrupamento, Dados de Cronometragem, centra-se na entidade REGISTRO\_TEMPO, que armazena todas as medições realizadas. A decisão de criar uma entidade independente para os registros de tempo, ao invés de vinculá-los diretamente à entidade CORRIDA, fundamenta-se em três necessidades principais: permitir múltiplos registros por equipe (para casos de tentativas ou validações), suportar o sistema híbrido de registro (manual com possibilidade de correção), e facilitar a auditoria através do rastreamento de quem registrou, quando registrou, e se o tempo foi validado. Cada REGISTRO\_TEMPO relaciona-se com a EQUIPE que realizou a descida, com a CORRIDA específica, e com o USUÁRIO responsável pelo registro, criando trilha completa de auditoria. O quinto agrupamento, Controle de Acesso, representa os USUÁRIOS do sistema (organizadores, cronometristas e demais operadores), vinculando-os aos registros de tempo que validaram ou corrigiram, garantindo rastreabilidade das operações críticas. As cardinalidades estabelecidas refletem as regras de negócio do evento: um EVENTO possui múltiplas EDIÇÕES (1:N), cada EDIÇÃO envolve múltiplas EQUIPES (1:N) e organiza-se em múltiplas BATERIAS (1:N), cada BATERIA contém múltiplas CORRIDAS (1:N), e cada CORRIDA gera múltiplos REGISTROS\_TEMPO (1:N). O relacionamento N:N entre EQUIPE e REGISTRO\_TEMPO permite que uma equipe participe de múltiplas corridas ao longo do evento.

\subsection{Arquitetura Proposta do Sistema}
\label{subsec:arquitetura-proposta}

A arquitetura do sistema foi concebida seguindo o padrão de camadas, amplamente reconhecido na engenharia de software por promover separação de responsabilidades, facilitar manutenção e permitir evolução independente de componentes (PRESSMAN; MAXIM, 2016). A solução proposta organiza-se em quatro camadas principais, conforme será detalhado na Figura Z na seção de Prototipação. A Camada de Apresentação (Frontend) será desenvolvida como aplicação web responsiva, garantindo acesso via navegadores desktop e dispositivos móveis. A interface deve adaptar-se aos diferentes perfis de usuário, apresentando dashboards simplificados para cronometristas (foco em agilidade), painéis administrativos completos para organizadores, e visualizações públicas otimizadas para acompanhamento dos resultados. A escolha por aplicação web, ao invés de aplicativos nativos, justifica-se pela necessidade de manutenção centralizada e pela dispensa de instalação por parte dos usuários, reduzindo barreiras de adoção. A Camada de Lógica de Negócio (Backend) concentrará todas as regras de negócio do evento, incluindo validação de inscrições, cálculo de classificações, geração de chaveamentos, e controle de permissões de usuários. Esta camada apresentará APIs RESTful para comunicação com o frontend, seguindo princípios de arquitetura orientada a serviços que facilitam escalabilidade e integração futura com outros sistemas. A separação clara entre frontend e backend permite que diferentes interfaces possam consumir os mesmos serviços, possibilitando futuramente, por exemplo, a criação de aplicativo mobile nativo sem alterações na lógica de negócio. A Camada de Persistência (Banco de Dados) será responsável pelo armazenamento confiável e eficiente de todos os dados do sistema. A estratégia de evolução gradual prevê inicialmente a utilização de SQLite durante as fases de desenvolvimento e testes, migrando posteriormente para PostgreSQL quando necessidades de concorrência e volume de dados justificarem banco de dados cliente-servidor robusto. Esta abordagem permite validar o modelo de dados com menor complexidade operacional, postergando investimentos em infraestrutura até o momento apropriado. A Camada de Comunicação implementará mecanismos de atualização em tempo real para o módulo de visualização de resultados. A utilização de WebSockets ou Server-Sent Events permitirá que alterações nos tempos e classificações sejam automaticamente propagadas para todos os clientes conectados, eliminando a necessidade de recarregamento manual das páginas e proporcionando experiência verdadeiramente dinâmica aos espectadores. Uma consideração arquitetural fundamental refere-se à resiliência do sistema frente às limitações de infraestrutura identificadas na fase de Imersão. A arquitetura contempla estratégias de funcionamento offline-first no frontend, permitindo que funcionalidades críticas como o registro de tempos continuem operacionais mesmo em caso de perda temporária de conectividade. Dados registrados offline serão armazenados localmente e sincronizados automaticamente quando a conexão for restabelecida, garantindo continuidade operacional do evento.

\subsection{Escolha de Tecnologias}
\label{subsec:escolha-tecnologias}

A seleção do conjunto tecnológico que materializará a solução proposta pautou-se por critérios de maturidade, suporte da comunidade, aderência aos requisitos identificados e capacitação da equipe de desenvolvimento. As escolhas foram realizadas através de análise comparativa de alternativas, considerando vantagens e limitações de cada opção. Para a camada de lógica de negócio, optou-se pela utilização do framework Spring Boot sobre a plataforma Java. Esta decisão fundamenta-se em diversos fatores convergentes. Primeiramente, Spring Boot oferece ecossistema maduro e abrangente para desenvolvimento de APIs RESTful, incluindo componentes prontos para autenticação (Spring Security), acesso a dados (Spring Data JPA), e exposição de serviços web (Spring Web). A curva de aprendizado da equipe, composta por estudantes familiarizados com Java através das disciplinas do curso, reduz riscos de projeto relacionados à capacitação técnica. Adicionalmente, Spring Boot promove convenção sobre configuração, reduzindo significativamente a quantidade de código boilerplate necessário e acelerando o desenvolvimento. A robustez e escalabilidade da plataforma Java garantem que o sistema poderá crescer conforme demandas futuras sem necessidade de reescrita completa. Por fim, a vasta documentação disponível e a comunidade ativa facilitam a resolução de problemas e a adoção de melhores práticas. Para a camada de apresentação, selecionou-se a biblioteca React como base para construção das interfaces. React destaca-se pela abordagem declarativa de construção de interfaces, onde componentes reutilizáveis encapsulam tanto lógica quanto apresentação, promovendo modularidade e manutenibilidade do código. A arquitetura baseada em componentes alinha-se perfeitamente com a necessidade de diferentes painéis para diferentes perfis de usuário, permitindo compartilhamento de elementos comuns enquanto especializa-se funcionalidades específicas. O ecossistema React oferece vasto conjunto de bibliotecas complementares para necessidades comuns: gerenciamento de estado (Redux ou Context API), rotas (React Router), comunicação em tempo real (Socket.io-client), e componentes de interface pré-construídos (Material-UI, Ant Design). A curva de aprendizado, embora presente, é suavizada pela abundância de recursos educacionais e pela crescente adoção da biblioteca no mercado. A estratégia de banco de dados prevê evolução em duas etapas. Inicialmente, SQLite será utilizado durante desenvolvimento e testes, aproveitando sua simplicidade de configuração (banco de dados em arquivo único) e adequação para cenários de concorrência limitada. Esta escolha permite à equipe focar no desenvolvimento da lógica de negócio e na validação do modelo de dados sem investir tempo em configuração e administração de servidor de banco de dados. Conforme o sistema amadurecer e aproximar-se de uso em produção, migração para PostgreSQL será realizada. PostgreSQL oferece recursos fundamentais para aplicações multi-usuário em produção: suporte robusto a transações ACID, controle de concorrência otimista, replicação para alta disponibilidade, e performance superior para operações de leitura/escrita simultâneas. A transição de SQLite para PostgreSQL será facilitada pela utilização de ORM (Object-Relational Mapping) através do Spring Data JPA, que abstrai detalhes específicos do banco de dados e permite mudança de provider com mínimas alterações no código. A ausência de rede WiFi estável na área da pista, identificada como limitação crítica na fase de Imersão, demandou pesquisa intensa visando uma solução arquitetural que garantisse comunicação em tempo real entre os dispositivos dos cronometristas, organizadores e público, sem depender de infraestrutura de internet externa. A solução adotada utiliza dois microcontroladores ESP32 configurados como Access Point (ponto de acesso WiFi) e servidores, criando uma rede autônoma à qual dispositivos móveis se conectam diretamente. O ESP32, além de suas capacidades de comunicação sem fio, possui recursos computacionais suficientes para hospedar servidor web leve e broker de mensagens, tornando-se o núcleo da infraestrutura de rede do evento. Dispositivos móveis (smartphones e tablets) dos usuários conectam-se à rede WiFi criada pelo ESP32, acessando a aplicação web servida localmente e recebendo atualizações em tempo real através de protocolo de mensageria. Para implementação da funcionalidade de ranking ao vivo e sincronização de dados entre múltiplos clientes conectados, avaliou-se três abordagens principais: polling periódico, WebSockets (FETTE; MELNIKOV, 2011), e protocolos de mensageria baseados em publish-subscribe como MQTT (BANKS; GUPTA, 2014) ou AMQP (Advanced Message Queuing Protocol, implementado pelo RabbitMQ) (OASIS, 2012). A técnica de polling, embora simples, gera tráfego desnecessário e introduz latência significativa na atualização, sendo descartada como opção viável. WebSockets oferecem comunicação bidirecional full-duplex eficiente (FETTE; MELNIKOV, 2011), porém demandam manutenção de conexões persistentes entre servidor e cada cliente, o que pode sobrecarregar o ESP32 quando múltiplos usuários estão conectados simultaneamente. Protocolos baseados em publish-subscribe, particularmente MQTT (Message Queuing Telemetry Transport), apresentam-se como solução mais adequada ao contexto do projeto. MQTT foi projetado especificamente para comunicação em dispositivos com recursos limitados e redes de baixa largura de banda (BANKS; GUPTA, 2014), características alinhadas com o cenário do ESP32 operando como servidor. O protocolo implementa padrão de mensageria onde publishers (publicadores) enviam mensagens para tópicos específicos, e subscribers (assinantes) recebem automaticamente todas as mensagens dos tópicos aos quais estão inscritos (HOHPE; WOOLF, 2003), sem necessidade de conexões diretas ponto-a-ponto. Esta arquitetura reduz significativamente a carga sobre o servidor em comparação com WebSockets, pois o broker MQTT gerencia eficientemente a distribuição de mensagens. Alternativamente, RabbitMQ implementa protocolo AMQP, oferecendo recursos mais avançados de roteamento de mensagens, garantias de entrega e persistência. Embora mais robusto, RabbitMQ demanda maior consumo de recursos computacionais e memória, potencialmente excedendo as capacidades do ESP32 quando operando simultaneamente como Access Point, servidor web e broker de mensagens. A decisão final entre MQTT e implementação simplificada baseada em AMQP será definida durante a fase de Prototipação, baseada em testes de carga que avaliarão o comportamento do ESP32 sob condições realistas de uso (estimados 50-100 dispositivos conectados simultaneamente durante o evento). O backend da aplicação, desenvolvido em Spring Boot, integrará cliente MQTT através da biblioteca Eclipse Paho ou Spring Integration MQTT, estabelecendo ponte entre as operações de banco de dados (persistência de registros de tempo) e a distribuição de mensagens para os clientes conectados. Quando o cronometrista registra tempo através da interface web, o backend persiste dados no banco local, valida o registro, e publica mensagem MQTT notificando todos os clientes assinantes, que atualizam suas interfaces imediatamente sem necessidade de recarregamento manual.

\subsection{Considerações sobre Tecnologias Não Adotadas}
\label{subsec:tecnologias-nao-adotadas}

A decisão de utilizar microcontrolador ESP32 como infraestrutura de rede fundamenta-se em análise pragmática que equilibra inovação tecnológica, viabilidade de implementação e alinhamento com os objetivos centrais do projeto. É fundamental distinguir entre dois usos potenciais de IoT (MINERVA; BIRU; ROTONDI, 2015) no contexto deste trabalho: (1) sensores automáticos de cronometragem, e (2) infraestrutura de comunicação e rede. Enquanto o primeiro foi deliberadamente descartado, o segundo constitui elemento essencial da arquitetura proposta. O ESP32, configurado como Access Point, cria rede WiFi autônoma com alcance suficiente para cobrir área do evento (estimados 100-150 metros lineares da pista). Sua capacidade de processar requisições HTTP, servir conteúdo estático (arquivos HTML/CSS/JavaScript da aplicação frontend), e operar broker MQTT simultaneamente, embora exija otimizações cuidadosas, está dentro das especificações técnicas do hardware. Arquitetura híbrida também foi considerada: múltiplos ESP32 distribuídos geograficamente (largada, meio da pista, chegada), cada um criando Access Point independente mas sincronizados através de rede mesh (AKYILDIZ et al., 2005) ou backbone cabeado. Esta abordagem aumenta a complexidade mas oferece redundância e melhor cobertura. A decisão entre arquitetura centralizada (um único ESP32) ou distribuída (múltiplos ESP32) será baseada em testes de campo que irão mensurar a qualidade do sinal nas diferentes posições da pista. Importante ressaltar que esta decisão não descarta evolução futura para cronometragem automática. Esta flexibilidade arquitetural permite que o projeto inicie com escopo gerenciável, valide conceitos em produção real, e evolua incrementalmente (PRESSMAN; MAXIM, 2016) conforme recursos e maturidade técnica da equipe permitirem.

\section{Prototipação}
\label{sec:prototipacao}

Xxxxxxxxxxxxxxxxxxxxx xxxxxxxxx xxxxxxxx xxxxxxxxxx xxxxxxxx. X xxxxxxxxxx xxxxxxxxxxx xxxxxxxxx xxxxxxxx xxxxxxxx xxxxxxxx x xxxxxxxxxx. Xxxxxxxxxxxxxxxxxxxxx xxxxxxxxx xxxxxxxx xxxxxxxxxx xxxxxxxx. X xxxxxxxxxx
